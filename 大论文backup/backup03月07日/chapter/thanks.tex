% !Mode:: "TeX:UTF-8"
%%%%%%% 以下内容不要修改!!! mzhy55
\makeatletter
\fancypagestyle{plain}{%
  \fancyhf{}%
  \renewcommand{\headrulewidth}{0pt}%
  \renewcommand{\footrulewidth}{0pt}%
%  \renewcommand{\headrule}{}
  \fancyhead[CE]{{\zihao{5} 燕山大学\CAST@value@degree 学位论文}}
  \fancyhead[CO]{\zihao{5} 致\ \ 谢}
  \fancyfoot[C]{{\zihao{-5} -~\thepage~-}}
  }
  \pagestyle{fancy}%%%%% 页眉 mzhy55
  \fancyhf{}
  \fancyhead[CE]{{\zihao{5} 燕山大学\CAST@value@degree 学位论文}}
  \fancyhead[CO]{{\zihao{5} 致\ \ 谢}}
  \fancyfoot[C]{{\zihao{-5} -~\thepage~-}}
\makeatother
%%%%%%% 以上内容不要修改!!! mzhy55

\begin{thanks}
时光荏苒,岁月如梭,这是我在燕山大学的第三个年头 
首先衷心感谢导师×××教授对本人的精心指导。他的言传身教将使我终生受益。

感谢刘志新教授,以及实验室全体老师和同窗们的热情帮助和支持!
 谢元艾师兄苏佳伟师兄李亚苹师姐高磊 赵松晗 孟祥云 张心哲 金小曹 齐峻霄 田秋来 陈熙 李晨生 祝犇 张嘉元 高杰 李彩月 梁自强 仵元梓
 郑晓阳
 李博 王天泽 王天骄 王志越 张宏林 孙震岳 召辉 曹海洋 陈勇
 雷军魏建军尹同跃李书福王传福余承东
本课题承蒙××××基金资助,特此致谢。
感谢评审专家抽出宝贵的时间
…
最后,感谢所有参加论文评审的各位专家,您们的宝贵意见对我今后研究工作
的开展具有重要的指导意义。

李亚杰老师的指导一定是最适合我的选择。初次见面是刚刚入学时李老师担任我们的辅导员,大二时的电路与信号系统是由李老师和桂老师讲授,再到后来的认知实习与毕业设计,每一年都和李亚杰老师有交集。初次接触此类课题是在cctv监控中心实验室老师带领我们认知实习,老师提到了行政楼依靠传感器的自动门系统,这个系统主要由两部分组成:发射器和接收器。发射器发射出的红外线被接收器接受,如果有人站在门前挡住了发射器发给接收器的红外线信号,那么就会有一个开门的信号传输给控制开门的电机把门打开,经过一段延时后,电机再通过传动把门关闭。自动门系统的缺点就是发射器发射出的信号是一个扇面儿的红外线被接收器接收,所以不仅会响应人站在门前把门打开,而且如果是猫狗等小动物或是没有生命的物体都会使门打开。也讲到了每年毕业设计会有学生选择做生命探测仪和智能IC卡门禁系统。从那时起我便确立了这些就是我有一定基础并极为感兴趣的毕设题目。今年的特殊原因使得我们需要在家完成毕业设计,但老师对我们的尽心尽责使我们并没有落下进度,他每周不管有多忙都会抽出一天的时间通过电话的方式帮我们解决遇到的问题并提出相应的修改意见,在老师的帮助下,论文一步步完善,这与老师兢兢业业的工作态度是离不开的,师者,所以传道受业解惑也,对老师给予我的传道授业表示感谢。
老师在学校传授知识,父母在家牵挂我们,随着跟父母在一起的时间越来越少,随着年龄的增长,也越来越能体会到父母的艰辛,正是因为有了他们无私的付出和支持,我才能走出不悔的人生路,我想:父母恩情,定将以他日成功予以回报。

\end{thanks}
