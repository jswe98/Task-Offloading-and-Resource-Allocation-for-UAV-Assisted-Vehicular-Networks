% !Mode:: "TeX:UTF-8"
%%%%%%% 以下内容不要修改!!! mzhy55
\makeatletter
\fancypagestyle{plain}{%
  \fancyhf{}%
  \renewcommand{\headrulewidth}{0pt}%
  \renewcommand{\footrulewidth}{0pt}%
%  \renewcommand{\headrule}{}
  \fancyhead[CE]{{\zihao{5} 燕山大学\CAST@value@degree 学位论文}}
  \fancyhead[CO]{\zihao{5} 结\ \ 论}
  \fancyfoot[C]{{\zihao{-5} -~\thepage~-}}
  }
  \pagestyle{fancy}%%%%% 页眉 mzhy55
  \fancyhf{}
  \fancyhead[CE]{{\zihao{5} 燕山大学\CAST@value@degree 学位论文}}
  \fancyhead[CO]{{\zihao{5} 结\ \ 论}}
  \fancyfoot[C]{{\zihao{-5} -~\thepage~-}}
\makeatother
%%%%%%% 以上内容不要修改!!! mzhy55

\begin{conclusion} \label{chap:conclusion}
%\chapter{结论} \label{chap:conclusion}
\textcolor[RGB]{18,20,168}{}本文以移动无人机辅助的云边计算的车联网为背景,充分体现了车辆用户的移动性,分别研究了空地一体化的大规模通信异构车载网络、
云辅助 MEC的车辆任务卸载网络、天地一体化的无人机辅助的车辆任务卸载通信网络三个
场景,在考虑了功率约束、无人机移动性约束、车辆用户服务质量约束等条件下,以吞吐量、能量效率为指标,对中继选择、无人机轨迹、功
率控制进行联合优化,通过博弈论、拉格朗日法、SCA 法、交替优化法、贝恩斯坦近似法、积分变换法等方法提升
车联网的高效性与可靠性。本文的研究工作可以总结为:

首先,针对空地一体化的大规模通信异构车载网络,提出了一种基于博弈的鲁
棒资源分配算法,该方案以用户间的博弈关系为核心,制定了实时功率分配和定价
策略,在新颖的优化方案中实现了用户利益的最大化。引入了概率约束,以确保用
户服务的可靠性和稳定性。仿真结果表明,所提算法具有复杂多用户干扰和信道不
确定性的空地一体化异构车载通信场景下是有效的。

其次,针对车辆网络越来越高的低延迟高数据计算的需求,提出了云辅助 MEC
的鲁棒功率控制和任务卸载的新方法。由于信道存在不确定性,优化问题受到传输
速率、计算通信延迟和同信道干扰概率形式的限制。最初的优化问题被表述为鲁棒
性功率控制和任务卸载调度问题,应用了 SCA 技术,将变量耦合的 NP 难问题转化
为可处理的凸问题。仿真结果表明,我们提出的算法得到了近似最优解。与现有方
法相比,系统平均卸载效用得到显著改善。

最后,考虑了更加实际的物理场景,将无人机辅助通信与任务卸载相结合,提
出了一种高效的天地一体化的无人机辅助双向车道的车辆通信方案。构建了车辆通
信时的吞吐量与通信及无人机飞行能耗的基本平衡方案。通过优化车辆的发射功率
与无人机的飞行轨迹,以及时隙的分配,可以使得系统的能效最大化,数值仿真表
明,该方案在能效方面的性能明显高于其他方法并可显著提升车联网通信效率。

但是现有工作需要进一步完善,主要包括以下几点:

(1) 现阶段所研究的场景中,构建的问题是高动态的车联网,但是优化过程很难获取到全局的信息,导致没有站在长期的视角进行优化。

(2) 只对上行链路通信研究了车辆用户的吞吐量以及能效的优化问题,并且认为节点以单工模式工作,并未考虑双工模式下的双向通信以及未来车辆通信过程中可能会遇到的窃听者窃听的问题。
当用户的信息在传输过程中面临被窃听的风险时,其信息安全便无法得到保障。因此,对车联网中的通信安全性能进行深入分析显得尤为重要,这也是该领域值得研究的关键所在。

(3) 此外,本文已经对车联网中的鲁棒功率控制以及车对与信道复用的资源优化问题进行了初步的探索。
然而,目前的研究主要停留在理论层面,未来的研究重点将转向构建实验平台,实现理论与实践的有机结合,
以期将最新的科研成果应用于实际,为相关领域的进步做出贡献。
\begin{comment}
结论应是作者在学位论文研究过程中所取得的创新性成果的概要总结,不能与摘要混为一谈。结论应包括论文的主要结果、创新点、展望三部分,在结论中应概括论文的核心观点,明确、客观地指出本研究内容的创新性成果(含新见解、新观点、方法创新、技术创新、理论创新),并指出今后进一步在本研究方向进行研究工作的展望与设想。对所取得的创新性成果应注意从定性和定量两方面给出科学、准确的评价,分(1)、(2)、(3)…条列出,宜用“提出了”、“建立了”等词叙述。此外,结论的撰写还应符合以下基本要求:

(1) 结论具有相对的独立性,不应是对论文中各章小结的简单重复。结论要与引言相呼应,以自身的条理性、明确性、客观性反映论文价值。对论文创新内容的概括,评价要适当。
(2) 结论措辞要准确、严谨,不能模棱两可,避免使用“大概”、“或许”、“可能是”等词语。结论中不应有解释性词语,而应直接给出结果。结论中一般不使用量的符号,而宜用量的名称。

(3) 结论应指出论文研究工作的局限性或遗留问题,如条件所限,或存在例外情况,或本论文尚难以解释或解决的问题。

(4) 常识性的结果或重复他人的结果不应作为结论。

技术难点:
(1)在各复现室外测试场景条件下对设计的分布式可靠传输策略、动态功率优化算法、以及鲁棒博弈策略是否适合于真是的物理环境。
(2)如何将制定的优化问题通过一定数学处理使得问题易于求解是个难点,以及如何得到有效的功率迭代算法是关键的问题。
(3)在进行仿真验证时,相关参数的选取会对结果产生重要影响,如何快速准确的设置相关的参数是仿真中面临的一个关键问题。。
创新点:
(1)考虑了车联网场景中由车辆高速移动所引起的信道不确定性,引入一阶马尔可夫过程。构建了合理可行的车联网络场景,使之在描述车联网动态特性的情况下,又能够通过相应的约束条件和目标函数保证网络通信服务质量。
(2)改进并推广了贝恩斯坦近似方法,将其运用于中断概率的矩阵形式中以处理大规模动态车辆网络环境下非凸的信干噪比约束。
(3)联合考虑了高动态车联网环境下的云边协同计算资源分配与功率优化,使得系统容量最大化的同时以最优卸载策略使车辆中的计算资源得到充分利用。

\end{comment}
\end{conclusion}
