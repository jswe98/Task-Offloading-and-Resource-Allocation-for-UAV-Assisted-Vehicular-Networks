% !Mode:: "TeX:UTF-8"
%%%%%%% 以下内容不要修改!!! mzhy55
\makeatletter
\fancypagestyle{plain}{%
  \fancyhf{}%
  \renewcommand{\headrulewidth}{0pt}%
  \renewcommand{\footrulewidth}{0pt}%
%  \renewcommand{\headrule}{}
  \fancyhead[CE]{{\zihao{5} 燕山大学\CAST@value@degree 学位论文}}
  \fancyhead[CO]{\zihao{5} 结\ \ 论}
  \fancyfoot[C]{{\zihao{-5} -~\thepage~-}}
  }
  \pagestyle{fancy}%%%%% 页眉 mzhy55
  \fancyhf{}
  \fancyhead[CE]{{\zihao{5} 燕山大学\CAST@value@degree 学位论文}}
  \fancyhead[CO]{{\zihao{5} 结\ \ 论}}
  \fancyfoot[C]{{\zihao{-5} -~\thepage~-}}
\makeatother
%%%%%%% 以上内容不要修改!!! mzhy55

\begin{conclusion} \label{chap:conclusion}
\textcolor[RGB]{18,20,168}{待完善我们联
车联网,新能源汽车行业上半场是电气化,下半场智能化,
结论作为学位论文正文的组成部分,单独排写,不加章标题序号,不标注引用文献。结论内容一般在2000字以内。}


结论应是作者在学位论文研究过程中所取得的创新性成果的概要总结,不能与摘要混为一谈。结论应包括论文的主要结果、创新点、展望三部分,在结论中应概括论文的核心观点,明确、客观地指出本研究内容的创新性成果(含新见解、新观点、方法创新、技术创新、理论创新),并指出今后进一步在本研究方向进行研究工作的展望与设想。对所取得的创新性成果应注意从定性和定量两方面给出科学、准确的评价,分(1)、(2)、(3)…条列出,宜用“提出了”、“建立了”等词叙述。此外,结论的撰写还应符合以下基本要求:

(1) 结论具有相对的独立性,不应是对论文中各章小结的简单重复。结论要与引言相呼应,以自身的条理性、明确性、客观性反映论文价值。对论文创新内容的概括,评价要适当。
(2) 结论措辞要准确、严谨,不能模棱两可,避免使用“大概”、“或许”、“可能是”等词语。结论中不应有解释性词语,而应直接给出结果。结论中一般不使用量的符号,而宜用量的名称。

(3) 结论应指出论文研究工作的局限性或遗留问题,如条件所限,或存在例外情况,或本论文尚难以解释或解决的问题。

(4) 常识性的结果或重复他人的结果不应作为结论。

技术难点:
(1)在各复现室外测试场景条件下对设计的分布式可靠传输策略、动态功率优化算法、以及鲁棒博弈策略是否适合于真是的物理环境。
(2)如何将制定的优化问题通过一定数学处理使得问题易于求解是个难点,以及如何得到有效的功率迭代算法是关键的问题。
(3)在进行仿真验证时,相关参数的选取会对结果产生重要影响,如何快速准确的设置相关的参数是仿真中面临的一个关键问题。。
创新点:
(1)考虑了车联网场景中由车辆高速移动所引起的信道不确定性,引入一阶马尔可夫过程。构建了合理可行的车联网络场景,使之在描述车联网动态特性的情况下,又能够通过相应的约束条件和目标函数保证网络通信服务质量。
(2)改进并推广了贝恩斯坦近似方法,将其运用于中断概率的矩阵形式中以处理大规模动态车辆网络环境下非凸的信干噪比约束。
(3)联合考虑了高动态车联网环境下的云边协同计算资源分配与功率优化,使得系统容量最大化的同时以最优卸载策略使车辆中的计算资源得到充分利用。


\end{conclusion}
