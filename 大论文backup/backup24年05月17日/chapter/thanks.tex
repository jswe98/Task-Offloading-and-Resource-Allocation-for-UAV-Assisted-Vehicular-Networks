% !Mode:: "TeX:UTF-8"
%%%%%%% 以下内容不要修改!!! mzhy55
\makeatletter
\fancypagestyle{plain}{%
  \fancyhf{}%
  \renewcommand{\headrulewidth}{0pt}%
  \renewcommand{\footrulewidth}{0pt}%
%  \renewcommand{\headrule}{}
  \fancyhead[CE]{{\zihao{5} 燕山大学\CAST@value@degree 学位论文}}
  \fancyhead[CO]{\zihao{5} 致\ \ 谢}
  \fancyfoot[C]{{\zihao{-5} -~\thepage~-}}
  }
  \pagestyle{fancy}%%%%% 页眉 mzhy55
  \fancyhf{}
  \fancyhead[CE]{{\zihao{5} 燕山大学\CAST@value@degree 学位论文}}
  \fancyhead[CO]{{\zihao{5} 致\ \ 谢}}
  \fancyfoot[C]{{\zihao{-5} -~\thepage~-}}
\makeatother
%%%%%%% 以上内容不要修改!!! mzhy55

\begin{thanks}
\begin{comment}
\end{comment}
时光荏苒,岁月如梭,这是我在燕山大学的第三个年头,转眼间我的硕士生涯已接近尾声。回首这段充满挑战与收获的旅程,我深感自己的每一步成长都离不开身边人的支持与帮助。在此,我谨以最诚挚的心意,向所有在我学术道路上给予关心、指导和帮助的人表示衷心的感谢。

首先,我要衷心感谢我的导师刘志新教授。刘老师治学严谨的科研态度,令我终身难忘。
从刚开始的迷茫,到取得后来的进步,从入门到毕业,从最初对于自己课题的不自
信到后来的成绩,都离不开刘老师的言传身教、循循善诱。刘老师的培养,不仅体
现在学术上,更对我的人生观和价值感也有着深刻的影响,在今后的工作学习和生
活中,我将铭记老师的深刻教诲,戒骄戒躁,严谨踏实,精益求精。

 感谢博士师兄苏佳伟、谢元艾、高磊、赵松晗、孟祥云、金小曹、齐峻霄,师姐 李亚萍 带我科研入门,帮助我解决了很多学术上的困难。
 感谢张心哲师兄  和刘子健师兄对我秋招找工作的提供了宝贵的经验。
 感谢师兄田秋来 、陈熙、李晨生、祝犇,师姐张嘉元、高杰对我生活上的关心和照顾,
 感谢同窗李彩月 、李博、梁自强、仵元梓 、郑晓阳和舍友王天泽、 王天骄 、王志越  在科研生活中的陪伴和帮助,%舍友王天泽、 王天骄 、王志越
 感谢张宏林、孙震、麻梓炀、岳召辉、曹海洋、陈勇等师弟师妹给我带来的欢乐。

 感谢燕山大学电气工程学院所有给我授课的老师,是他们耐心的讲解让我对专业课程有了深入的了解与研究。感谢所有参与开题报告、中期答辩以及论文评审的
老师们,感谢他们的宝贵意见让论文不断完善。师者,所以传道受业解惑也,对老师给予我的传道授业表示感谢。
老师在学校传授知识,父母在家牵挂我们,随着跟父母在一起的时间越来越少,随着年龄的增长,也越来越能体会到父母的艰辛,正是因为有了他们无私的付出和支持,我才能走出不悔的人生路,我想:父母恩情,定将以他日成功予以回报。

最后,感谢所有参加论文评审的各位专家,您们的宝贵意见对我今后研究工作
的开展具有重要的指导意义。

\begin{comment}
李亚杰老师的指导一定是最适合我的选择。初次见面是刚刚入学时李老师担任我们的辅导员,大二时的电路与信号系统是由李老师和桂老师讲授,再到后来的认知实习与毕业设计,每一年都和李亚杰老师有交集。初次接触此类课题是在cctv监控中心实验室老师带领我们认知实习,老师提到了行政楼依靠传感器的自动门系统,这个系统主要由两部分组成:发射器和接收器。发射器发射出的红外线被接收器接受,如果有人站在门前挡住了发射器发给接收器的红外线信号,那么就会有一个开门的信号传输给控制开门的电机把门打开,经过一段延时后,电机再通过传动把门关闭。自动门系统的缺点就是发射器发射出的信号是一个扇面儿的红外线被接收器接收,所以不仅会响应人站在门前把门打开,而且如果是猫狗等小动物或是没有生命的物体都会使门打开。也讲到了每年毕业设计会有学生选择做生命探测仪和智能IC卡门禁系统。从那时起我便确立了这些就是我有一定基础并极为感兴趣的毕设题目。今年的特殊原因使得我们需要在家完成毕业设计,但老师对我们的尽心尽责使我们并没有落下进度,他每周不管有多忙都会抽出一天的时间通过电话的方式帮我们解决遇到的问题并提出相应的修改意见,在老师的帮助下,论文一步步完善,这与老师兢兢业业的工作态度是离不开的,师者,所以传道受业解惑也,对老师给予我的传道授业表示感谢。
老师在学校传授知识,父母在家牵挂我们,随着跟父母在一起的时间越来越少,随着年龄的增长,也越来越能体会到父母的艰辛,正是因为有了他们无私的付出和支持,我才能走出不悔的人生路,我想:父母恩情,定将以他日成功予以回报。
\end{comment}
\end{thanks}
