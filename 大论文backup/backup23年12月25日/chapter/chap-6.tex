% !Mode:: "TeX:UTF-8"
\chapter{数字物理量与单位}
\label{chap:table}

\section{普通三线表}\label{section3-1}
科技文献中常用的三线表:
\begin{table}[htbp!]
 \centering\small
 \Tablecaption{燕山大学硕士学位论文参考文献规则}\label{tab:ysubof}
\begin{tabular}{llr}
 \toprule
    论文版本    & 参考文献标准    & 实施年份(年)  \\
 \midrule
    旧版        & BF7714-87       & 1987            \\
    新版        & GBT7714-2005    & 2005            \\
 \bottomrule
 \end{tabular}
\end{table}

实现代码如下:
\begin{verbatim}
\begin{table}[htbp!]
 \centering\small
 \Tablecaption{燕山大学硕士学位论文参考文献规则}\label{tab:ysubof}
\begin{tabular}{llr}
 \toprule
    论文版本    & 参考文献标准    & 实施年份(年)  \\
 \midrule
    旧版        & BF7714-87       & 1987            \\
    新版        & GBT7714-2005    & 2005            \\
 \bottomrule
 \end{tabular}
\end{table}
\end{verbatim}

\section{有合并列的三线表}\label{section3-2}
合并列通常见于表格的第一行,在适当的位置使用\verb|\multicolumn| 命令即可。
\begin{table}[htbp!]
\centering\small
\Tablecaption{带有合并列的三线表}\label{tab:test}
\begin{tabular}{llr} \toprule
\multicolumn{2}{c}{Item} \\ \cmidrule(r){1-2}
Animal & Description & Price (\$)\\ \midrule
Gnat & per gram & 13.65 \\
& each & 0.01 \\
Gnu & stuffed & 92.50 \\
Emu & stuffed & 33.33 \\
Armadillo & frozen & 8.99 \\ \bottomrule
\end{tabular}
\end{table}


该表格是采用如下代码实现的:
\begin{verbatim}
\begin{table}[htbp!]
\centering\small
\Tablecaption{带有合并列的三线表}\label{tab:test}
\begin{tabular}{llr} \toprule
\multicolumn{2}{c}{Item} \\ \cmidrule(r){1-2}
Animal & Description & Price (\$)\\ \midrule
Gnat & per gram & 13.65 \\
& each & 0.01 \\
Gnu & stuffed & 92.50 \\
Emu & stuffed & 33.33 \\
Armadillo & frozen & 8.99 \\ \bottomrule
\end{tabular}
\end{table}
\end{verbatim}


\section{特殊形式的表格}\label{section3-3}
\begin{verbatim}
\begin{table}[htbp!]
	\centering\small
	\Tablecaption{The relation of $E({{L}_{q}})$ with ${{p}_{2}}$
    and $\theta$}\label{tab.2}
	\begin{tabular*}{\columnwidth}{@{\extracolsep{\fill}}@{~~}cccccccc@{~~}}
		\toprule
		\multicolumn{7}{c}{ \hspace{2cm} The expected waiting queue length
         $E({{L}_{q}})$}\\\cline{2-8}
		\raisebox{1ex}[0pt]{$\theta$}  &$p_2=0.1$     &$p_2=0.15$  &$p_2=0.2$
        &$p_2=0.25$ &$p_2=0.3$  &$p_2=0.35$   &$p_2=0.4$\\
		\midrule
		0.3     &16.4830  &5.1232   &2.9232   &1.9704   &1.4339   &1.0886   &0.8479\\
		0.5     &9.0488   &3.7848   &2.2906   &1.5839   &1.1723   &1.9035   &0.7146 \\
		0.7     &7.4321   &3.3256   &2.0528   &1.4338   &1.0686   &0.8291   &0.6607 \\
		\bottomrule
	\end{tabular*}	
\end{table}
\end{verbatim}
生成
\begin{table}[htbp!]
	\centering\small
	\Tablecaption{The relation of $E({{L}_{q}})$ with ${{p}_{2}}$ and $\theta$}\label{tab.2}
	\begin{tabular*}{\columnwidth}{@{\extracolsep{\fill}}@{~~}cccccccc@{~~}}
		\toprule
		\multicolumn{7}{c}{ \hspace{2cm} The expected waiting queue length $E({{L}_{q}})$}\\
		\cline{2-8}
		\raisebox{1ex}[0pt]{$\theta$}  &$p_2=0.1$     &$p_2=0.15$  &$p_2=0.2$   &$p_2=0.25$
        &$p_2=0.3$  &$p_2=0.35$   &$p_2=0.4$\\
		\midrule
		0.3     &16.4830  &5.1232   &2.9232   &1.9704   &1.4339   &1.0886   &0.8479\\
		0.5     &9.0488   &3.7848   &2.2906   &1.5839   &1.1723   &1.9035   &0.7146 \\
		0.7     &7.4321   &3.3256   &2.0528   &1.4338   &1.0686   &0.8291   &0.6607 \\
		\bottomrule
	\end{tabular*}	
\end{table}

\section{表题的生成}\label{section3-4}
其实上边的例子中已经包含了表题的引用命令\verb|\Tablecaption|。
例如表\ref{tab:ysubof}中:
\begin{verbatim}
\Tablecaption{燕山大学硕士学位论文参考文献规则}\label{tab:ysubof}
\end{verbatim}
为当前的表格添加中文图题“燕山大学硕士学位论文参考文献规则”。同时添加标签“tab:ysubof”。 对表格的引用就是通过标签来实现的。

\section{表格的引用}\label{section3-5}
表格的引用同样是使用\verb|\ref{}| 命令实现的。例如“表\verb|\ref{tab:ysubof}|” 输出的结果为:表\ref{tab:ysubof}。\LaTeX 会自动将其替换为表格的编号。例如:
\begin{verbatim}
燕山大学硕士学位论文参考文献规则的表格如表\ref{tab:ysubof}所示。
\end{verbatim}
的效果如下:\\
燕山大学硕士学位论文参考文献规则的表格如表\ref{tab:ysubof}所示。

\section{本章小结}\label{section3-5}
注意!从第二章开始应有``本章小结",主要总结本章所做的主要研究工作,研究成果等内容!!!

%




\label{chap:unit}
模板加载了siunitx 宏包,可以实现长串数字位数的正确分割和各种物理量单位的自动格式化,避免
手工调用数学环境输入单位。尤其适用于理工科各种物理量的输入。该宏包的引入主要是为了解决论文
格式标准中的这个要求:
数字的书写不必每格一个数码,一般每两数码占一格,数字间分节不用分位号",",\emph{凡4位或4位以上的数都从个位起每3位数空\textbf{半个数码(1/4汉字)}。“\num{3 000000}”,不要写成}“3,000,000”,\emph{小数点后的数从小数点起向右按\textbf{每三位一组分节}。一个用阿拉伯数字书写的多位数不能从数字中间转行。}

\section{数字}\label{section7-1}
使用\verb|\num| 命令可以输入正确格式的长数字,包括科学计数法格式的数字。

\begin{table}[htbp]
\centering\zihao{5}
\Tablecaption{siunitx 宏包与\LaTeX 数学环境输出效果对比}
\begin{tabular}{ll|ll}
\toprule
siunitx 输出样式    & siunitx 输入方式          & \LaTeX 数学环境输出样式  & \LaTeX 数学环境输入方式     \\
\midrule
\num{123456789}     & \verb|\num{123456789}|    & 123456789             & \verb|123456789|         \\
\num{-1000000}      & \verb|\num{-1000000}|     & $-1000000$            & \verb|$-1000000$|        \\
\num{3.2e-8}        & \verb|\num{3.2e-8}|       & $3.2\times 10^{-8}$   & \verb|$3.2\times 10^{-8}$|\\
\num{1.2345678}     & \verb|\num{1.2345678}|    & 1.2345678             & \verb|1.2345678|          \\
\bottomrule
\end{tabular}
\end{table}


\section{单位}\label{section7-2}
单独输入单位时,可以采用\verb|\si|命令。

\begin{table}[htbp]
\centering\zihao{5}
\Tablecaption{单位的不同输入方式}
\begin{tabular}{ll}
\toprule
输出样式  &输入方式     \\
\midrule
\si{kg.m/s^2}                           & \verb|\si{kg.m/s^2}|         \\
\si{g_{polymer}mol_{cat}.s^{-1}}       & \verb|\si{g_{polymer}mol_{cat}.s^{-1}}|\\
\si{\kilo\gram\metre\per\square\second} & \verb|\si{\kilo\gram\metre\per\square\second}|\\
\si{\gram\per\cubic\centi\metre}        &\verb|\si{\gram\per\cubic\centi\metre}|\\
\si{\square\volt\cubic\lumen\per\farad} &\verb|\si{\square\volt\cubic\lumen\per\farad}|\\
\si{\metre\squared\per\gray\cubic\lux}  &\verb|\si{\metre\squared\per\gray\cubic\lux}|\\
\si{\henry\second}                      &\verb|\si{\henry\second}|\\
\bottomrule
\end{tabular}
\end{table}

\section{同时输入数字与单位}\label{section7-3}

通常情况下,数字与单位是共同给出的,这时可以采用\verb|\SI| 命令。注意这里的 SI 是大写的。并且加入不同的可选项,最终的效果也不同。

\begin{table}[htbp]
\centering\zihao{5}
\Tablecaption{同时输入数字与单位}
\begin{tabular}{ll}
\toprule
输出样式    & 输入方式  \\
\midrule
\SI[mode=text]{1.23}{J.mol^{-1}.K^{-1}}         & \verb|\SI[mode=text]{1.23}{J.mol^{-1}.K^{-1}}| \\
\SI{.23e7}{\candela}                            & \verb|\SI{.23e7}{\candela}|\\
\SI[per-mode=symbol]{1.99}[\$]{\per\kilogram}   & \verb|\SI[per-mode=symbol]{1.99}[\$]{\per\kilogram}|\\
\SI[per-mode=fraction]{1,345}{\coulomb\per\mole}& \verb|\SI[per-mode=fraction]{1,345}{\coulomb\per\mole}|\\
\bottomrule
\end{tabular}
\end{table}

\section{附1:国际标准单位与导出单位输入方式}\label{section7-4}

\begin{table}[htbp]
\centering\zihao{5}
\Tablecaption{国际标准单位输入方式}
\begin{tabular}{lllp{10pt}lll}
\toprule
单位    & 命令  & 符号  &   & 单位    & 命令  & 符号  \\
\midrule
安培    & \verb|\ampere|    & \si{\ampere}   && 坎德拉  & \verb|\candela|   & \si{\candela}  \\
开尔文  & \verb|\kelvin|    & \si{\kelvin}   && 千克    & \verb|\kilogram|  & \si{\kilogram}    \\
米      & \verb|\meter|     & \si{\meter}    && 摩尔    & \verb|\mole|      & \si{\mole}    \\
秒      & \verb|\second|    & \si{\second}   &&     &        &      \\
\bottomrule
\end{tabular}
\end{table}

\begin{table}[htbp]
\centering\zihao{5}
\Tablecaption{国际标准导出单位输入方式}
\begin{tabular}{lllp{10pt}lll}
\toprule
单位    & 命令  & 符号  &   & 单位    & 命令  & 符号  \\
\midrule
becquerel       & \verb|\becquerel|     & \si{\becquerel}       & & newton      & \verb|\newton|    & \si{\newton} \\
degree Celsius  & \verb|\degreeCelsius| & \si{\degreeCelsius}   & & ohm         & \verb|\ohm|       & \si{\ohm}\\
coulomb         & \verb|\coulomb|       & \si{\coulomb}         & & pascal      & \verb|\pascal|    & \si{\pascal}\\
farad           & \verb|\farad|         & \si{\farad}           & & radian      & \verb|\radian|    & \si{\radian} \\
gray            & \verb|\gray|          & \si{\gray}            & & siemens     & \verb|\siemens|   & \si{\siemens}    \\
hertz           & \verb|\hertz|         & \si{\hertz}           & & sievert     & \verb|\sievert|   & \si{\sievert}\\
henry           & \verb|\henry|         & \si{\henry}           & & steradian   & \verb|\steradian| & \si{\steradian}\\
joule           & \verb|\joule|         & \si{\joule}           & & tesla       & \verb|\tesla|     & \si{\tesla}\\
katal           & \verb|\katal|         & \si{\katal}           & & volt        & \verb|\volt|      & \si{\volt}\\
lumen           & \verb|\lumen|         & \si{\lumen}           & & watt        & \verb|\watt|      & \si{\watt}\\
lux             & \verb|\lux|           & \si{\lux}             & & weber       & \verb|\weber|     & \si{\weber}\\
\bottomrule
\end{tabular}
\end{table}
%\section{附2:国际标准单位前缀}\label{section7-5}
%
%\begin{table}[htbp]
%\centering\zihao{5}
%\caption{国际标准单位前缀输入方式}
%\begin{tabular}{llllp{10pt}llll}
%\toprule
%名称    & 命令          & 符号          & 指数  &   & 名称      & 命令          & 符号      & 指数 \\
%\midrule
%yocto   & \verb|\yocto| & \si{\yocto}   & -24   &   & deca      &\verb|\deca|   &\si{\deca} & 1     \\
%zepto   & \verb|\zepto| & \si{\zepto}   & -21   &   & hecto     &\verb|\hecto|  &\si{\hecto}& 2\\
%atto    & \verb|\atto|  & \si{\atto}    & -18   &   & kilo      &\verb|\kilo|   &\si{\kilo} & 3\\
%femto   & \verb|\femto| & \si{\femto}   & -15   &   & mega      &\verb|\mega|   &\si{\mega} & 6\\
%pico    & \verb|\pico|  & \si{\pico}    & -12   &   & giga      &\verb|\giga|   &\si{\giga} & 9\\
%nano    & \verb|\nano|  & \si{\nano}    & -9    &   & tera      &\verb|\tera|   &\si{\tera} & 12\\
%micro   & \verb|\micro| & \si{\micro}   & -6    &   & peta      &\verb|\peta|   &\si{\peta} & 15\\
%milli   & \verb|\milli| & \si{\milli}   & -3    &   & exa       &\verb|\exa|    &\si{\exa}  & 18\\
%centi   & \verb|\centi| & \si{\centi}   & -2    &   & zetta     &\verb|\zetta|  &\si{\zetta}& 21\\
%deci    & \verb|\deci|  & \si{\deci}    & -1    &   & yotta     &\verb|\yotta|  &\si{\yotta}& 24\\
%\bottomrule
%\end{tabular}
%\end{table}


\section{本章小结}\label{section7-5}
注意!从第二章开始应有``本章小结",主要总结本章所做的主要研究工作,研究成果等内容!!!


%
