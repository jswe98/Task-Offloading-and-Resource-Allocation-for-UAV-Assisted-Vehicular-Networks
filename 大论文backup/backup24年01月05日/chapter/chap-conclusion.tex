% !Mode:: "TeX:UTF-8"
%%%%%%% 以下内容不要修改!!! mzhy55
\makeatletter
\fancypagestyle{plain}{%
  \fancyhf{}%
  \renewcommand{\headrulewidth}{0pt}%
  \renewcommand{\footrulewidth}{0pt}%
%  \renewcommand{\headrule}{}
  \fancyhead[CE]{{\zihao{5} 燕山大学\CAST@value@degree 学位论文}}
  \fancyhead[CO]{\zihao{5} 结\ \ 论}
  \fancyfoot[C]{{\zihao{-5} -~\thepage~-}}
  }
  \pagestyle{fancy}%%%%% 页眉 mzhy55
  \fancyhf{}
  \fancyhead[CE]{{\zihao{5} 燕山大学\CAST@value@degree 学位论文}}
  \fancyhead[CO]{{\zihao{5} 结\ \ 论}}
  \fancyfoot[C]{{\zihao{-5} -~\thepage~-}}
\makeatother
%%%%%%% 以上内容不要修改!!! mzhy55

\begin{conclusion}
\label{chap:conclusion}

结论作为学位论文正文的组成部分,单独排写,不加章标题序号,不标注引用文献。结论内容一般在2000字以内。

结论应是作者在学位论文研究过程中所取得的创新性成果的概要总结,不能与摘要混为一谈。结论应包括论文的主要结果、创新点、展望三部分,在结论中应概括论文的核心观点,明确、客观地指出本研究内容的创新性成果(含新见解、新观点、方法创新、技术创新、理论创新),并指出今后进一步在本研究方向进行研究工作的展望与设想。对所取得的创新性成果应注意从定性和定量两方面给出科学、准确的评价,分(1)、(2)、(3)…条列出,宜用“提出了”、“建立了”等词叙述。此外,结论的撰写还应符合以下基本要求:

(1) 结论具有相对的独立性,不应是对论文中各章小结的简单重复。结论要与引言相呼应,以自身的条理性、明确性、客观性反映论文价值。对论文创新内容的概括,评价要适当。

(2) 结论措辞要准确、严谨,不能模棱两可,避免使用“大概”、“或许”、“可能是”等词语。结论中不应有解释性词语,而应直接给出结果。结论中一般不使用量的符号,而宜用量的名称。

(3) 结论应指出论文研究工作的局限性或遗留问题,如条件所限,或存在例外情况,或本论文尚难以解释或解决的问题。

(4) 常识性的结果或重复他人的结果不应作为结论。

\end{conclusion}
