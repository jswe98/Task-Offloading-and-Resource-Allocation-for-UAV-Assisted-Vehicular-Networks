% !Mode:: "TeX:UTF-8"
\chapter{参考文献}
\label{chap:equ}
本章介绍基本公式的输入方法;矩阵和向量的输入;方程组的输入;多行公式的换行与对齐。
\LaTeX 中数学公式的输入依赖于数学环境。

在正文中用到的简短公式,可以直接使用两个美元符号“\$”括起来,如:
\begin{verbatim}
直角三角形三边长度满足关系式$a^2+b^2=c^2$。
\end{verbatim}
得到的结果是:\\
直角三角形三边长度满足关系式$a^2+b^2=c^2$。\\

而对于一些较为重要或者较复杂、需要编号的公式,则需要使用各种数学环境,例如使用equation环境:
\begin{verbatim}
\begin{equation}\label{chp-mode}
\mathbfit{E}=\mathrm{Re}(\mathbfit{E}(\mathbfit{r}))e^{j\omega t}.
\end{equation}
\end{verbatim}
得到的结果是:
\begin{equation}\label{chp-mode}
\mathbfit{E}=\mathrm{Re}(\mathbfit{E}(\mathbfit{r}))e^{j\omega t}.
\end{equation}
对它的引用方式为:\verb|公式\eqref{chp-mode}|。\\
得到的结果为:公式\eqref{chp-mode}。
\begin{equation}\label{E24}
\begin{array}{ll}
\textrm{Pr}\left\{\frac{1}{\tau_iR_i-k_i}+\frac{c_{i,e}}{f_i}\le D_{max}\right\}\\
=\textrm{Pr}\left\{R_i\geq\frac{1}{R_i\left(D_{max}-D_2\right)}+\frac{k_i}{\tau_i}\right\}\\
\!\le1\!-\!\textrm{Pr}\left\{p_i{\widetilde{g}}_{i,j}^k\le\left(I_{th}+\sigma^2\right)2^\frac{1+k_i\left(D_{max}-D_2\right)}{\tau_i\left(D_{max}-D_2\right)}-p_i{\hat{g}}_{i,j}^k\right\}\\
=\!1\!-\!\int_{0}^{\left(I_{th}+\sigma^2\right)2^\frac{1+k_i\left(D_{max}-D_2\right)}{\tau_i\left(D_{max}-D_2\right)}-p_i{\hat{g}}_{i,j}^k}{e^{-x}dx}\!\geq\!1-\varepsilon_2.
\end{array}
\end{equation}
如果不想对公式进行编号,则可以使用equation*环境:
\begin{verbatim}
\begin{equation*}\label{chp-m2}
\mathbfit{E}=\mathrm{Re}(\mathbfit{E}(\mathbfit{r}))e^{j\omega t}.
\end{equation*}
\end{verbatim}
得到的结果是:
\begin{equation*}\label{chp-m2}
\mathbfit{E}=\mathrm{Re}(\mathbfit{E}(\mathbfit{r}))e^{j\omega t}.
\end{equation*}

\section{上下标}\label{section4-1}
\verb|a_1+b^2\times c_1^2=0|输出结果为:$a_1+b^2\times c_1^2=0$。

\section{分式}\label{section4-2}
命令\verb|\frac, \dfrac\ tfrac|可以用来输出分式:
\begin{verbatim}
\begin{equation}\label{fr}
  \sin\dfrac{\cos\dfrac{a}{b}}{c}=
  \sin\frac{\cos\frac{a}{b}}{c}=
  \sin\tfrac{\cos\tfrac{a}{b}}{c}.
\end{equation}
\end{verbatim}
输出的结果是:
\begin{equation}\label{fr}
\sin\dfrac{\cos\dfrac{a}{b}}{c}=
\sin\frac{\cos\frac{a}{b}}{c}=
\sin\tfrac{\cos\tfrac{a}{b}}{c}.
\end{equation}

当使用括号来括起纵向尺寸较大的对象例如分式时,要使用\verb|\left| 和
\verb|\right| 命令使括号在纵向上伸长。例如:
\begin{verbatim}
\begin{equation}\label{frr}
  \left(\frac{a}{b}\right)=(\frac{a}{b}).
\end{equation}
\end{verbatim}
的输出结果是:
\begin{equation}\label{frr}
\left(\frac{a}{b}\right)=(\frac{a}{b}).
\end{equation}

\section{矢量点乘与叉乘}\label{section4-3}
矢量点乘:\verb|$\mathbfit{A}\cdot\mathbfit{B}$|输出:$\mathbfit{A}\cdot\mathbfit{B}$。

矢量叉乘:\verb|$\mathbfit{C}\times\mathbfit{D}$|输出:$\mathbfit{C}\times\mathbfit{D}$。

\section{求和与积分}\label{section4-4}
命令\verb|\sum|和命令\verb|\int|负责输出求和与积分号。例如:
\begin{verbatim}
\begin{equation}\label{equ-sum}
  \sum_{i=1}^n\sin\beta_i^2=0.
\end{equation}
\end{verbatim}
输出结果为:
\begin{equation}\label{equ-sum}
\sum_{i=1}^n\sin\beta_i^2=0.
\end{equation}
\begin{verbatim}
\begin{equation}\label{equ-int}
  \int_a^b\frac{c}{d}\,\mathrm{d}x=0.
\end{equation}
\end{verbatim}
输出结果为:
\begin{equation}\label{equ-int}
  \int_a^b\frac{c}{d}\,\mathrm{d}x=0.
\end{equation}


\section{矩阵与数组}\label{section4-5}
矩阵与数组使用array环境:
\begin{verbatim}
\begin{equation}\label{equ-array}
  \left(
    \begin{array}{c} a \\ c \end{array}
  \right)=
  \left(
    \begin{array}{cc} a & b \\ c & d \end{array}
  \right).
\end{equation}
\end{verbatim}
输出结果是:
\begin{equation}\label{equ-array}
  \left(
  \begin{array}{c} a \\ c \end{array}
  \right)=
  \left(
  \begin{array}{cc} a & b \\ c & d \end{array}
  \right).
\end{equation}
也可以使用matrix环境:
\begin{verbatim}
\begin{equation}\label{equ-matrix}
  \begin{matrix} 0 & 1 \\ 1 & 0\end{matrix}=
  \begin{pmatrix}0 &-i \\ i & 0\end{pmatrix}=
  \begin{bmatrix}1 & 0 \\ 0 &-1\end{bmatrix}=
  \begin{vmatrix}a & b \\ c & d\end{vmatrix}.
\end{equation}
\end{verbatim}
输出结果是:
\begin{equation}\label{equ-matrix}
  \begin{matrix} 0 & 1 \\ 1 & 0\end{matrix}=
  \begin{pmatrix}0 &-i \\ i & 0\end{pmatrix}=
  \begin{bmatrix}1 & 0 \\ 0 &-1\end{bmatrix}=
  \begin{vmatrix}a & b \\ c & d\end{vmatrix}.
\end{equation}

\section{多行公式与对齐方法}\label{section4-6}
多行公式排列,每个公式都有自己的编号通常使用align环境。例如:
\begin{verbatim}
\begin{align}
  a_1+a_2+a_3 &=0, \label{equ-s1}\\
  b_1+b_2+b_3+b_4 &=0, \label{equ-s2}\\
  c_1+c_2 &=0. \label{equ-v1}
\end{align}
\end{verbatim}
输出结果为:
\begin{align}
  a_1+a_2+a_3 &=0, \label{equ-s1}\\
  b_1+b_2+b_3+b_4 &=0, \label{equ-s2}\\
  c_1+c_2 &=0. \label{equ-v1}
\end{align}
其中符号“\&”为对齐符号。这里实现了等号对齐。

也可以使用eqnarray环境输入
\begin{verbatim}
\begin{eqnarray}
  a_1+a_2+a_3 &=&0, \label{equ-s1a}\\
  b_1+b_2+b_3+b_4 &=&0, \label{equ-s2a}\\
  c_1+c_2 &=&0. \label{equ-v1a}
\end{eqnarray}
\end{verbatim}
得到效果如下:
\begin{eqnarray}
  a_1+a_2+a_3 &=&0, \label{equ-s1a}\\
  b_1+b_2+b_3+b_4 &=&0, \label{equ-s2a}\\
  c_1+c_2 &=&0. \label{equ-v1a}
\end{eqnarray}

\section{带有大括号的方程组}\label{section4-7}
与多行公式不同,方程组左侧使用“\verb|\left{|”加了一个大括号,另外只有一个公式编号,因此采用equation和aligned结合的方式,例如:
\begin{verbatim}
\begin{equation}\label{equ-fml}
  \left\{
  \begin{aligned}
    x^2+y^2 &=0,\\
    x+y+z^2 &=0,\\
    x^2+y+z &=0.
  \end{aligned}
  \right.
\end{equation}
\end{verbatim}
输出结果为:
\begin{equation}\label{equ-fml}
  \left\{
  \begin{aligned}
    x^2+y^2 &=0,\\
    x+y+z^2 &=0,\\
    x^2+y+z &=0.
  \end{aligned}
  \right.
\end{equation}
也可以这样对齐,这样输入:
\begin{verbatim}
\begin{equation}\label{equ-fmla}
  \left\{
  \begin{array}{l}
    x^2+y^2 =0,\\
    x+y+z^2 =0,\\
    x^2+y+z =0.
  \end{array}
  \right.
\end{equation}
\end{verbatim}
得到的结果如下:
\begin{equation}\label{equ-fmla}
  \left\{
  \begin{array}{l}
    x^2+y^2 =0,\\
    x+y+z^2 =0,\\
    x^2+y+z =0.
  \end{array}
  \right.
\end{equation}

\section{特殊的公式}\label{section4-8}
\begin{verbatim}
$$
\bordermatrix{
& 0 & 1 & 2\cr
0 & A & B & C\cr
1 & d & e & f\cr
2 & 1 & 2 & 3},
$$
\begin{equation}
\bordermatrix{&a_1&a_2&...&a_n\cr
          b_1 & 1.2  & 3.3  & 5.1  & 2.8  \cr
          c_1 & 4.7  & 7.8  & 2.4  & 1.9  \cr
          ... & ...  & ...  & ...  & ...  \cr
          z_1 & 8.0  & 9.9  & 0.9  & 9.99  \cr}.
\end{equation}
\end{verbatim}
输出结果为:
$$
\bordermatrix{
& 0 & 1 & 2\cr
0 & A & B & C\cr
1 & d & e & f\cr
2 & 1 & 2 & 3},
$$
\begin{equation}
\bordermatrix{&a_1&a_2&...&a_n\cr
          b_1 & 1.2  & 3.3  & 5.1  & 2.8  \cr
          c_1 & 4.7  & 7.8  & 2.4  & 1.9  \cr
          ... & ...  & ...  & ...  & ...  \cr
          z_1 & 8.0  & 9.9  & 0.9  & 9.99  \cr}.
\end{equation}

\section{数学环境的使用}\label{section4-9}


一些常见的数学环境:
\begin{verbatim}
\theoremstyle{plain}
  \newtheorem{algo}{算法~}[chapter]
  \newtheorem{thm}{定理~}[chapter]
  \newtheorem{lem}[thm]{引理~}
  \newtheorem{prop}[thm]{命题~}
  \newtheorem{cor}[thm]{推论~}
\theoremstyle{definition}
  \newtheorem{defn}{定义~}[chapter]
  \newtheorem{conj}{猜想~}[chapter]
  \newtheorem{exmp}{例~}[chapter]
  \newtheorem{rem}{注~}
  \newtheorem{case}{情形~}
\theoremstyle{break}
  \newtheorem{bthm}[thm]{定理~}
  \newtheorem{blem}[thm]{引理~}
  \newtheorem{bprop}[thm]{命题~}
  \newtheorem{bcor}[thm]{推论~}
\renewcommand{\proofname}{\bf 证明}
\end{verbatim}

\subsection{定义}\label{defn}
\begin{verbatim}
\begin{defn}\label{defn4-1}
对于一些较为重要或者较复杂、需要编号的公式,则需要使用各种数学环境,例如使用equation环境:
得到的结果是:
\begin{equation}
\mathbfit{E}=\mathrm{Re}(\mathbfit{E}(\mathbfit{r}))e^{j\omega t}.
\end{equation}
  对它的引用方式为:\verb|公式\eqref{chp-mode}|,
得到的结果为:公式\eqref{chp-mode}。
\end{defn}
\end{verbatim}

\begin{defn}\label{defn4-1}
对于一些较为重要或者较复杂、需要编号的公式,则需要使用各种数学环境,例如使用equation环境:
得到的结果是:
\begin{equation}
\mathbfit{E}=\mathrm{Re}(\mathbfit{E}(\mathbfit{r}))e^{j\omega t}.
\end{equation}

对它的引用方式为:\verb|公式\eqref{chp-mode}|,
得到的结果为:公式\eqref{chp-mode}。
\end{defn}

\subsection{定理及证明}\label{thm}
\begin{verbatim}
\begin{thm}\label{theorem4-1}
对于一些较为重要或者较复杂、需要编号的公式,则需要使用各种数学环境,例如使用equation环境:得到的结果是:
\begin{equation}
\mathbfit{E}=\mathrm{Re}(\mathbfit{E}(\mathbfit{r}))e^{j\omega t}.
\end{equation}
  对它的引用方式为:\verb|公式\eqref{chp-mode}|,
得到的结果为:公式\eqref{chp-mode}。
\end{thm}
\begin{proof}
对于一些较为重要或者较复杂、需要编号的公式,则需要使用各种数学环境,例如使用equation环境。
\end{proof}
\end{verbatim}

\begin{thm}\label{theorem4-1}
对于一些较为重要或者较复杂、需要编号的公式,则需要使用各种数学环境,例如使用equation环境:得到的结果是:
\begin{equation}
\mathbfit{E}=\mathrm{Re}(\mathbfit{E}(\mathbfit{r}))e^{j\omega t}.
\end{equation}

对它的引用方式为:\verb|公式\eqref{chp-mode}|,
得到的结果为:公式\eqref{chp-mode}。
\end{thm}
\begin{proof}
对于一些较为重要或者较复杂、需要编号的公式,则需要使用各种数学环境,例如使用equation环境。
\end{proof}

\subsection{推论}\label{cor}
\begin{verbatim}
\begin{cor}\label{cor4-2}
推论推论推论推论推论推论推论:
对它的引用方式为:
得到的结果为:公式\eqref{chp-mode}。
\end{cor}
\end{verbatim}

\begin{cor}\label{cor4-2}
推论推论推论推论推论推论推论:
对它的引用方式为:
得到的结果为:公式\eqref{chp-mode}.
\end{cor}

\subsection{引理}\label{lem}
\begin{verbatim}
\begin{lem}\label{lemma4-2}
引理引理引理引理引理引理引理:对于一些较为重要或者较复杂、需要编号的公式,则需要使用各种数学环境,例如使用equation 环境:
\end{lem}
\end{verbatim}
\begin{lem}\label{lemma4-2}
引理引理引理引理引理引理引理:对于一些较为重要或者较复杂、需要编号的公式,则需要使用各种数学环境,例如使用equation 环境:
\end{lem}

\subsection{例子}\label{exmp}
\begin{verbatim}
\begin{exmp}\label{exmp4-2}
例例例例例例例例例例例例例例:
得到的结果是:
\begin{equation}
\mathbfit{E}=\mathrm{Re}(\mathbfit{E}(\mathbfit{r}))e^{j\omega t}.
\end{equation}
对它的引用方式为:\verb|公式\eqref{chp-mode}|,
得到的结果为:公式\eqref{chp-mode}。
\end{exmp}
\end{verbatim}
\begin{exmp}\label{exmp4-2}
例例例例例例例例例例例例例例:
得到的结果是:
\begin{equation}
\mathbfit{E}=\mathrm{Re}(\mathbfit{E}(\mathbfit{r}))e^{j\omega t}.
\end{equation}

对它的引用方式为:\verb|公式\eqref{chp-mode}|,
得到的结果为:公式\eqref{chp-mode}。
\end{exmp}

\section{本章小结}\label{section4-10}
注意!从第二章开始应有``本章小结",主要总结本章所做的主要研究工作,研究成果等内容!!!

%


\label{chap:bib}

所有被引用的参考文献信息均存储在模板目录中的“bib”目录下,文件名为“tex.bib”。
由于使用了BibTeX,参考文献的格式是不需要手动调整。模板中的ysubst.bst 文件负责
文献格式输出。这里推荐您使用软件JabRef 来对文献进行管理。JabRef 支持中文,它可
以在这个网址下载到。\url{http://jabref.sourceforge.net/}

首先需要了解一个概念叫“BibTeX key”,也称作“BibTeX键”。它可以简单的理解为一篇
参考文献的“身份证号”。每一篇参考文献均有一个属于自己的不会重复的BibTeX key。
在引用文献的时候,需要使用引用命令\verb|\supercite{}|以上角标形式出现,使用引用命令\verb|\cite{}| 以正文形式出现。

\section{参考文献在正文和列表中格式要求}

\subsection{在正文中的要求}

在正文中,引用参考文献格式要求如下。

如果是单个作者的文献,应该写成:Kivshar\supercite{Kivshar2008}研究了...,得到了...结果。

如果是多个作者的文献,应该写成:Knight等\supercite{liu2021}研究了...,得到了...结果。
姚建铨等\supercite{yaojianquan2009}研究了...,得到了...结果。只需要列出第一个作者加上``等''字即可。中英文的文献是一样的标准,都是加上`` 等''字。

文献的序号紧接着名字出现,而不是放到这句话的最后面,大家要统一一下。

\subsection{在列表中的要求}

参考文献的格式一定按照bib文件夹里的tex文件的格式书写,特别指出:

(1) 文献标识一定要加上,比如J、M、C、D等。

(2) 当三个以上作者是省略掉,只列出前三个,后边加上``等''、``et al''。

(3) 期刊的年、卷、期、页的格式:2018, 30(4): 15-30. 当缺少卷或期时,在参考文献tex.bib的文件里把相应的项不填就可以了。

上述指出的问题只要是严格按照模板格式去填写,就不会出现的。为了防止有上述不规范的情况出现,请打印装订之前务必检查是否有类似问题。


\section{单一参考文献}\label{section5-1}
例如这里我引用一篇文献:
\begin{verbatim}
Knight等\supercite{Knight1996}研究了...Russell是光子晶体光纤之父...。
\end{verbatim}
其中“Knight1996”是我要引用的文献的BibTeX 键。输出的结果为:

Knight等\supercite{Knight1996}研究了...Russell是光子晶体光纤之父...。

注意文献的编号是自动生成的,并且具有超链接功能。单击编号可以定位到文末的参考文献章节。

\section{多个连续参考文献}\label{section5-2}
 如果要一次引用多个文献,只要在引用命令中用英文逗号隔开各个BibTeX键即可,例如:
 \begin{verbatim}
我要引用2篇文献\supercite{Knight1996,Knight2000}。
\end{verbatim}
输出结果为:

我要引用2篇文献\supercite{Knight1996,Knight2000}。


如果是3 篇或者以上,加入更多BibTeX 键即可。例如:
 \begin{verbatim}
3篇文献\supercite{Knight1996,Knight2000,Knight2002}。
\end{verbatim}
输出结果为:

3篇文献\supercite{Knight1996,Knight2000,Knight2002}。

更多文献的例子:
 \begin{verbatim}
很多很多文献\supercite{Kivshar2008,John1987,Jing2010,Jeon2005,%
Jastrow2008,Jackson2008,Huttunen2005,Hou2008,Hilligsoe2004,%
Hassani2008,Han2002}。
\end{verbatim}

输出的结果为:很多很多文献\supercite{John1987,Jing2010,Jeon2005,%
Jastrow2008,Jackson2008,Huttunen2005,Hou2008,Hilligsoe2004,%
Hassani2008,Han2002}。



\section{多个不连续参考文献}\label{section5-3}
 \begin{verbatim}
四个不连续文献\supercite{Zhu2004,Zhu2001,Han2002,Knight1996}。
\end{verbatim}
四个不连续文献\supercite{Zhu2004,Zhu2001,Han2002,Knight1996}。

专利及学位论文格式如下:
\cite{jxz,Yu2004,Zhao2013}。

多个作者的中文文献显示如下\cite{yaojianquan2009},language=\{Chinese\}很重要,在bib文件中,中文文献要加上!会议论文的格式如下\cite{Li1998}。


\section{本章小结}\label{section5-4}
注意!从第二章开始应有``本章小结",主要总结本章所做的主要研究工作,研究成果等内容!!!
