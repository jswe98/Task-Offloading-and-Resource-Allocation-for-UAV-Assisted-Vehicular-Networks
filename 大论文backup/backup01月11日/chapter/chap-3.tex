% !Mode:: "TeX:UTF-8"
\chapter{云辅助的车辆网络功率控制与任务卸载}
\label{chap:table}

\section{引言}\label{section3-1}
\label{chap:introduction}
云辅助移动边缘计算(C-MEC)为车载网络提供了丰富的计算资源,是一种前景广阔的任务卸载解决方案。本文提出了一种稳健的功率
控制和任务卸载方案,以卸载计算任务并最大化 C-MEC 网络的效用。然而,不确定的信道状态会严重影响卸载任务的传输稳定性。为了模拟信道的不确定性,采用了一阶马尔可夫过程,并考虑了车辆的移动性。此外,由于频谱资源有限,假设信道重用会导致复杂的同信道干扰。为了克服这些限制,对信号链路实施了概率约束,以确保通信质量。采用伯恩斯坦近似法将原始约束转化为可解约束。此外,还进一步采用了块坐标下降(BCD)方法和连续凸近似(SCA)技术来解决非凸鲁棒性优化问题。为确定最优解,提出了一种鲁棒电源控制和任务卸载调度算法。对提出的算法进行了数值模拟,以评估系统的性能。结果表明,与基准模型相比,该算法非常有效,尤其是在信道不确定的通信环境中。
移动边缘计算(MEC)和移动云计算(MCC)是新兴5G网络的两种新架构。
移动边缘计算(MEC)和移动云计算(MCC)作为新兴的5G网络的两种新架构,通常用于支持物联网设备的任务卸载、
特别是提供低延迟、高可靠性的计算服务。在网络中心的边缘,MEC可以减少传输延迟,并为车辆分配计算资源,以缓解计算压力
\cite{CCO}。
然而,当计算任务要求较高时,MEC的计算资源仍显不足。由于高性能计算由云服务器提供,基于云的计算网络已被部署以满足爆炸式增长的计算卸载需求。然而,云计算中心往往远离主干道,导致云计算延迟较长。
\cite{Qian2023}.
在高动态车联网中,车辆传输的数据必须实时处理。因此,在网络架构中部署C-MEC,以提供丰富的计算资源并减少传输延迟。
\section{系统模型与问题描述}\label{section3-2}
本文研究的C-MEC车载网络如图\ref{F1}所示,由MEC层和云计算层分层计算卸载架构组成。众多车辆在RSU的覆盖范围内被划分为多个地理区域,每个RSU下覆盖一个小区,每个RSU配备一台MEC服务器,为车辆提供计算卸载服务。我们将移动系统中的两组车辆和MEC服务器分别记为$\mathcal{V}=\left\{1,2,..., V\right\}$和$\mathcal{M}=\left\{1,2,..., M\right\}$。高速移动无线通信链路称为 V2RSU(V2R)链路,固定有线连接链路称为 RSU 到云(R2C)链路。详细的卸载过程描述如下。首先,车辆通过无线接口向云发送卸载请求信息,其中包括所需的通信资源、任务 ID 和提交时间,以及任务的最大可容忍服务时间。其次,MEC 服务器根据接收到的请求信息进行调度,包括任务上传服务器和任务计算服务器。最后,任务上传后,任务被推送到服务器队列中,直到服务器执行任务。此外,本文中使用的一些术语如表所示。
\begin{figure}[H]
\centering
\includegraphics[width=8cm]{figures//chap3//model22.eps}
\caption{System model.}
\label{F1}
\end{figure}

\begin{table}[!ht]
\caption{Notations}
\centering
{\small\begin{tabular}{ll}
\hline
\hline
\label{table:1}
$\textrm{Pr}\{\cdot\}$  & \!\!\!\!\!\!Probability function. \\
%$\mathbit{E}\left(\mathbit{x}\right)$ & Exponential distribution. \\
$\mathbb{R}^k$  & \!\!\!\!\!\!\!\!\! Set of $k$-dimensional real vectors.\\
$\mathbf{f}$  & \!\!\!\!\!\!\!Index set of computing resource $\mathbf{f}$$=$$[f_1,\cdots,f_i,\cdots,f_M]$. \\
$\mathbf{p}$  & \!\!\!\!\!\!\!\!\! Index set of vehicle power $\mathbf{p}$$=$$[p_1,\cdots,p_i,\cdots,p_M]$. \\
$\mathcal{M}$  & \!\!\!\!\!\!\!Index set of vehicles over a time slot $\mathcal{M}$$=$$\{1,2,\cdots,M\}$.\\
$\mathcal{V}$  & \!\!\!\!\!\!\!Index set of all active vehicles $\mathcal{V}$$=$$\{1,2,\cdots,V\}$.\\
$\textrm{E}\{\cdot\}$ & \!\!\!\!\!\!\!Expected value of a random variable. \\
%$\mathcal{J}$  & Index set of all CHs $\mathcal{J}$$=$$\{1,2,\cdots,N\}$. \\
\hline
\hline
\end{tabular}}
\end{table}
\subsection{通信模型}\label{section3-2-1}
由于车辆移动速度快,通信模式与传统的蜂窝通信不同。因此,很难直接获得 CSI。其中,RSU 仅能准确获取车辆到 RSU 链路的大尺度衰落 $L^2$,而小尺度衰落 $h$ 受多普勒效应引起的快速信道变化影响较大。我们假设CSI是通过信道估计获得的,因此,我们利用一阶高斯-马尔可夫过程\cite{Kim2011}对每个传输时间间隔内的小尺度衰落信道估计$h$建模如下、

\subsection{车辆计算模型}\label{section3-2-2}

\subsection{问题的定义}\label{section3-2-3}
\begin{table}[htbp!]
 \centering\small
 \Tablecaption{燕山大学硕士学位论文参考文献规则}\label{tab:ysubof}
\begin{tabular}{llr}
 \toprule
    论文版本    & 参考文献标准    & 实施年份(年)  \\
 \midrule
    旧版        & BF7714-87       & 1987            \\
    新版        & GBT7714-2005    & 2005            \\
 \bottomrule
 \end{tabular}
\end{table}

实现代码如下:
\begin{verbatim}
\begin{table}[htbp!]
 \centering\small
 \Tablecaption{燕山大学硕士学位论文参考文献规则}\label{tab:ysubof}
\begin{tabular}{llr}
 \toprule
    论文版本    & 参考文献标准    & 实施年份(年)  \\
 \midrule
    旧版        & BF7714-87       & 1987            \\
    新版        & GBT7714-2005    & 2005            \\
 \bottomrule
 \end{tabular}
\end{table}
\end{verbatim}

\section{问题的求解}\label{section3-3}
合并列通常见于表格的第一行,在适当的位置使用\verb|\multicolumn| 命令即可。
\subsection{目标方程中的连续凸逼近方法}\label{section3-3-1}
\begin{table}[htbp!]
\centering\small
\Tablecaption{带有合并列的三线表}\label{tab:test}
\begin{tabular}{llr} \toprule
\multicolumn{2}{c}{Item} \\ \cmidrule(r){1-2}
Animal & Description & Price (\$)\\ \midrule
Gnat & per gram & 13.65 \\
& each & 0.01 \\
Gnu & stuffed & 92.50 \\
Emu & stuffed & 33.33 \\
Armadillo & frozen & 8.99 \\ \bottomrule
\end{tabular}
\end{table}

\subsection{中断概率的近似}\label{section3-3-2}
该表格是采用如下代码实现的:
\begin{verbatim}
\begin{table}[htbp!]
\centering\small
\Tablecaption{带有合并列的三线表}\label{tab:test}
\begin{tabular}{llr} \toprule
\multicolumn{2}{c}{Item} \\ \cmidrule(r){1-2}
Animal & Description & Price (\$)\\ \midrule
Gnat & per gram & 13.65 \\
& each & 0.01 \\
Gnu & stuffed & 92.50 \\
Emu & stuffed & 33.33 \\
Armadillo & frozen & 8.99 \\ \bottomrule
\end{tabular}
\end{table}
\end{verbatim}
\subsection{优化功率问题}\label{section3-3-3}

\subsection{计算资源分配}\label{section3-3-4}

\section{仿真结果和性能分析}\label{section3-4}
\begin{verbatim}
\begin{table}[htbp!]
	\centering\small
	\Tablecaption{The relation of $E({{L}_{q}})$ with ${{p}_{2}}$
    and $\theta$}\label{tab.2}
	\begin{tabular*}{\columnwidth}{@{\extracolsep{\fill}}@{~~}cccccccc@{~~}}
		\toprule
		\multicolumn{7}{c}{ \hspace{2cm} The expected waiting queue length
         $E({{L}_{q}})$}\\\cline{2-8}
		\raisebox{1ex}[0pt]{$\theta$}  &$p_2=0.1$     &$p_2=0.15$  &$p_2=0.2$
        &$p_2=0.25$ &$p_2=0.3$  &$p_2=0.35$   &$p_2=0.4$\\
		\midrule
		0.3     &16.4830  &5.1232   &2.9232   &1.9704   &1.4339   &1.0886   &0.8479\\
		0.5     &9.0488   &3.7848   &2.2906   &1.5839   &1.1723   &1.9035   &0.7146 \\
		0.7     &7.4321   &3.3256   &2.0528   &1.4338   &1.0686   &0.8291   &0.6607 \\
		\bottomrule
	\end{tabular*}	
\end{table}
\end{verbatim}
生成
\begin{table}[htbp!]
	\centering\small
	\Tablecaption{The relation of $E({{L}_{q}})$ with ${{p}_{2}}$ and $\theta$}\label{tab.2}
	\begin{tabular*}{\columnwidth}{@{\extracolsep{\fill}}@{~~}cccccccc@{~~}}
		\toprule
		\multicolumn{7}{c}{ \hspace{2cm} The expected waiting queue length $E({{L}_{q}})$}\\
		\cline{2-8}
		\raisebox{1ex}[0pt]{$\theta$}  &$p_2=0.1$     &$p_2=0.15$  &$p_2=0.2$   &$p_2=0.25$
        &$p_2=0.3$  &$p_2=0.35$   &$p_2=0.4$\\
		\midrule
		0.3     &16.4830  &5.1232   &2.9232   &1.9704   &1.4339   &1.0886   &0.8479\\
		0.5     &9.0488   &3.7848   &2.2906   &1.5839   &1.1723   &1.9035   &0.7146 \\
		0.7     &7.4321   &3.3256   &2.0528   &1.4338   &1.0686   &0.8291   &0.6607 \\
		\bottomrule
	\end{tabular*}	
\end{table}

\section{本章小结}\label{section3-5}
在本章中,我们提出了一个新颖的鲁棒功率控制算法,优化方法针对于保证车辆通信的QoS的同时最大化效用。由于信道不确定性的存在,
其实上边的例子中已经包含了表题的引用命令
为当前的表格添加中文图题“燕山大学硕士学位论文参考文献规则”。同时添加标签“tab:ysubof”。 对表格的引用就是通过标签来实现的。
\begin{comment}
\section{表格的引用}\label{section3-6}
表格的引用同样是使用\verb|\ref{}| 命令实现的。例如“表\verb|\ref{tab:ysubof}|” 输出的结果为:表\ref{tab:ysubof}。\LaTeX 会自动将其替换为表格的编号。例如:
\begin{verbatim}
燕山大学硕士学位论文参考文献规则的表格如表\ref{tab:ysubof}所示。
\end{verbatim}
的效果如下:\\
燕山大学硕士学位论文参考文献规则的表格如表\ref{tab:ysubof}所示。

\section{本章小结}\label{section3-7}
注意!从第二章开始应有``本章小结",主要总结本章所做的主要研究工作,研究成果等内容!!!

\end{comment}

%
