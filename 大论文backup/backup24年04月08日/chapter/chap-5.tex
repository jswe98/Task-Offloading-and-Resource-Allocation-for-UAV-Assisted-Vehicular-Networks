% !Mode:: "TeX:UTF-8"
\chapter{结论} \label{chap:equcc}
\begin{comment}
首先,针对空地一体化的大规模通信异构车载网络,提出了一种基于博弈的鲁
棒资源分配算法,该方案以用户间的博弈关系为核心,制定了实时功率分配和定价
策略,在新颖的优化方案中实现了用户利益的最大化。引入了概率约束,以确保用
户服务的可靠性和稳定性。仿真结果表明,所提算法具有复杂多用户干扰和信道不
确定性的空地一体化异构车载通信场景下是有效的。

其次,针对车辆网络越来越高的低延迟高数据计算的需求,提出了云辅助 MEC
的鲁棒功率控制和任务卸载的新方法。由于信道存在不确定性,优化问题受到传输
速率、计算通信延迟和同信道干扰概率形式的限制。最初的优化问题被表述为鲁棒
性功率控制和任务卸载调度问题,应用了 SCA 技术,将变量耦合的 NP 难问题转化
为可处理的凸问题。仿真结果表明,我们提出的算法得到了近似最优解。与现有方
法相比,系统平均卸载效用得到显著改善。

最后,考虑了更加实际的物理场景,将无人机辅助通信与任务卸载相结合,提
出了一种高效的天地一体化的无人机辅助双向车道的车辆通信方案。构建了车辆通
信时的吞吐量与通信及无人机飞行能耗的基本平衡方案。通过优化车辆的发射功率
与无人机的飞行轨迹,以及时隙的分配,可以使得系统的能效最大化,数值仿真表
明,该方案在能效方面的性能明显高于其他方法并可显著提升车联网通信效率。
\end{comment}
本文研究了无人机辅助车联网边缘计算网络的功率控制与资源分配优化方案。将悬停的无人机辅助方案改进为
可轨迹规划的场景,并引入了车辆与路边单元计算与卸载的问题。本文的主要创新成果与结论可以归纳为以下三个方面:

首先,针对空地一体化的大规模通信异构车载网络,设计了一种基于博弈论的鲁棒资源分配算法。该算法的核心在于构建用户间的博弈关系,通过制定实时功率分配与定价策略,旨在实现用户利益的最大化。为确保服务的可靠性与稳定性,还引入了概率约束。仿真实验充分验证,在面临复杂多用户干扰与信道不确定性的挑战时,该算法在空地一体化异构车载通信场景中博弈出了最优解,为后续无人机辅助通信与任务卸载提供了技术支撑。

其次,针对车辆网络对低延迟和高数据计算能力的日益增长的需求,提出了一种云辅助移动边缘计算(C-MEC)鲁棒功率控制和任务卸载策略。考虑到信道的不确定性,优化问题受到传输速率、计算通信延迟以及同信道干扰概率等多重因素的制约。为了解决这一复杂问题,将原始的优化问题重构为了鲁棒性功率控制和任务卸载调度问题,并合理地运用了连续凸近似(SCA)技术。这一方法将原本变量耦合的NP难问题转化为易于处理的凸问题。仿真的结果显示,该算法能够得出近似最优解,与现有方法相比,系统的平均卸载效用得到了显著提升。

最后,为了进一步考虑无人机异构的物理场景,结合了前两章的研究,将无人机辅助通信与任务卸载融为一体,提出了一种空地一体化无人机辅助双向车道车辆通信方案。该方案致力于在车辆通信时实现吞吐量与通信及无人机飞行能耗之间的基本平衡。通过优化车辆的发射功率、无人机的飞行轨迹以及时隙的分配,成功实现了系统能效的最大化。数值仿真结果表明,这一方案在能效方面的表现显著优于其他方法,能够有效提升车联网的通信效率,为智能交通的发展提供了有力支持。

但是现有工作需要进一步完善,主要包括以下两点:

(1) 车辆用户的信息在传输过程中面临被窃听的风险时,其信息安全便无法得到保障。现阶段所研究的场景中,未能考虑车辆通信过程中可能会遇到的窃听者窃听的问题。
%构建的问题是高动态的车联网,但是优化过程很难获取到全局的信息,导致没有站在长期的视角进行优化。

%(2) 只对上行链路通信研究了车辆用户的吞吐量以及能效的优化问题,并且认为节点以单工模式工作,并未考虑双工模式下的双向通信以及未来车辆通信过程中可能会遇到的窃听者窃听的问题。当用户的信息在传输过程中面临被窃听的风险时,其信息安全便无法得到保障。因此,对车联网中的通信安全性能进行深入分析显得尤为重要,这也是该领域值得研究的关键所在。

(2) 此外,本文已经对车联网中的鲁棒功率控制以及车对与信道复用的资源优化问题进行了初步的探索。
然而,目前的研究主要停留在理论层面,未来的研究重点将转向构建实验平台,实现理论与实践的有机结合,
以期将最新的科研成果应用于实际,为相关领域的进步做出贡献。
