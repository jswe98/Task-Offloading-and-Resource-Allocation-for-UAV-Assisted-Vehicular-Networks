% !Mode:: "TeX:UTF-8"
\chapter{云辅助的车辆网络功率控制与任务卸载}
\label{chap:table}

\section{引言}\label{section3-1}
\label{chap:introduction}
移动边缘计算和云计算作为5G中的两个新兴技术,被越来越多的提出被用作于低延时高可靠性的物联网设备,尤其是车联网系统中。
边缘服务器位于网络的边缘,可以减少传输时延并且
这也是本模板要解决的问题。

科技文献中常用的三线表:
\section{系统模型与问题描述}\label{section3-2}
在本研究中,C-MEC车辆网络如图1所示

\subsection{通信模型}\label{section3-2-1}
由于车辆移动的快速性

\subsection{车辆计算模型}\label{section3-2-2}

\subsection{问题的定义}\label{section3-2-3}
\begin{table}[htbp!]
 \centering\small
 \Tablecaption{燕山大学硕士学位论文参考文献规则}\label{tab:ysubof}
\begin{tabular}{llr}
 \toprule
    论文版本    & 参考文献标准    & 实施年份(年)  \\
 \midrule
    旧版        & BF7714-87       & 1987            \\
    新版        & GBT7714-2005    & 2005            \\
 \bottomrule
 \end{tabular}
\end{table}

实现代码如下:
\begin{verbatim}
\begin{table}[htbp!]
 \centering\small
 \Tablecaption{燕山大学硕士学位论文参考文献规则}\label{tab:ysubof}
\begin{tabular}{llr}
 \toprule
    论文版本    & 参考文献标准    & 实施年份(年)  \\
 \midrule
    旧版        & BF7714-87       & 1987            \\
    新版        & GBT7714-2005    & 2005            \\
 \bottomrule
 \end{tabular}
\end{table}
\end{verbatim}

\section{问题的求解}\label{section3-3}
合并列通常见于表格的第一行,在适当的位置使用\verb|\multicolumn| 命令即可。
\subsection{目标方程中的连续凸逼近方法}\label{section3-3-1}
\begin{table}[htbp!]
\centering\small
\Tablecaption{带有合并列的三线表}\label{tab:test}
\begin{tabular}{llr} \toprule
\multicolumn{2}{c}{Item} \\ \cmidrule(r){1-2}
Animal & Description & Price (\$)\\ \midrule
Gnat & per gram & 13.65 \\
& each & 0.01 \\
Gnu & stuffed & 92.50 \\
Emu & stuffed & 33.33 \\
Armadillo & frozen & 8.99 \\ \bottomrule
\end{tabular}
\end{table}

\subsection{中断概率的近似}\label{section3-3-2}
该表格是采用如下代码实现的:
\begin{verbatim}
\begin{table}[htbp!]
\centering\small
\Tablecaption{带有合并列的三线表}\label{tab:test}
\begin{tabular}{llr} \toprule
\multicolumn{2}{c}{Item} \\ \cmidrule(r){1-2}
Animal & Description & Price (\$)\\ \midrule
Gnat & per gram & 13.65 \\
& each & 0.01 \\
Gnu & stuffed & 92.50 \\
Emu & stuffed & 33.33 \\
Armadillo & frozen & 8.99 \\ \bottomrule
\end{tabular}
\end{table}
\end{verbatim}
\subsection{优化功率问题}\label{section3-3-3}

\subsection{计算资源分配}\label{section3-3-4}

\section{仿真结果和性能分析}\label{section3-4}
\begin{verbatim}
\begin{table}[htbp!]
	\centering\small
	\Tablecaption{The relation of $E({{L}_{q}})$ with ${{p}_{2}}$
    and $\theta$}\label{tab.2}
	\begin{tabular*}{\columnwidth}{@{\extracolsep{\fill}}@{~~}cccccccc@{~~}}
		\toprule
		\multicolumn{7}{c}{ \hspace{2cm} The expected waiting queue length
         $E({{L}_{q}})$}\\\cline{2-8}
		\raisebox{1ex}[0pt]{$\theta$}  &$p_2=0.1$     &$p_2=0.15$  &$p_2=0.2$
        &$p_2=0.25$ &$p_2=0.3$  &$p_2=0.35$   &$p_2=0.4$\\
		\midrule
		0.3     &16.4830  &5.1232   &2.9232   &1.9704   &1.4339   &1.0886   &0.8479\\
		0.5     &9.0488   &3.7848   &2.2906   &1.5839   &1.1723   &1.9035   &0.7146 \\
		0.7     &7.4321   &3.3256   &2.0528   &1.4338   &1.0686   &0.8291   &0.6607 \\
		\bottomrule
	\end{tabular*}	
\end{table}
\end{verbatim}
生成
\begin{table}[htbp!]
	\centering\small
	\Tablecaption{The relation of $E({{L}_{q}})$ with ${{p}_{2}}$ and $\theta$}\label{tab.2}
	\begin{tabular*}{\columnwidth}{@{\extracolsep{\fill}}@{~~}cccccccc@{~~}}
		\toprule
		\multicolumn{7}{c}{ \hspace{2cm} The expected waiting queue length $E({{L}_{q}})$}\\
		\cline{2-8}
		\raisebox{1ex}[0pt]{$\theta$}  &$p_2=0.1$     &$p_2=0.15$  &$p_2=0.2$   &$p_2=0.25$
        &$p_2=0.3$  &$p_2=0.35$   &$p_2=0.4$\\
		\midrule
		0.3     &16.4830  &5.1232   &2.9232   &1.9704   &1.4339   &1.0886   &0.8479\\
		0.5     &9.0488   &3.7848   &2.2906   &1.5839   &1.1723   &1.9035   &0.7146 \\
		0.7     &7.4321   &3.3256   &2.0528   &1.4338   &1.0686   &0.8291   &0.6607 \\
		\bottomrule
	\end{tabular*}	
\end{table}

\section{本章小结}\label{section3-5}
在本章中,我们提出了一个新颖的鲁棒功率控制算法,优化方法针对于保证车辆通信的QoS的同时最大化效用。由于信道不确定性的存在,
其实上边的例子中已经包含了表题的引用命令\verb|\Tablecaption|。
例如表\ref{tab:ysubof}中:
\begin{verbatim}
\Tablecaption{燕山大学硕士学位论文参考文献规则}\label{tab:ysubof}
\end{verbatim}
为当前的表格添加中文图题“燕山大学硕士学位论文参考文献规则”。同时添加标签“tab:ysubof”。 对表格的引用就是通过标签来实现的。

\section{表格的引用}\label{section3-6}
表格的引用同样是使用\verb|\ref{}| 命令实现的。例如“表\verb|\ref{tab:ysubof}|” 输出的结果为:表\ref{tab:ysubof}。\LaTeX 会自动将其替换为表格的编号。例如:
\begin{verbatim}
燕山大学硕士学位论文参考文献规则的表格如表\ref{tab:ysubof}所示。
\end{verbatim}
的效果如下:\\
燕山大学硕士学位论文参考文献规则的表格如表\ref{tab:ysubof}所示。

\section{本章小结}\label{section3-7}
注意!从第二章开始应有``本章小结",主要总结本章所做的主要研究工作,研究成果等内容!!!

%
