% !Mode:: "TeX:UTF-8"
\makeatletter
\fancypagestyle{plain}{%
  \fancyhf{}%
  \renewcommand{\headrulewidth}{0pt}%
  \renewcommand{\footrulewidth}{0pt}%
%  \renewcommand{\headrule}{}
  \fancyhead[CE]{{\zihao{5} 燕山大学\CAST@value@degree 学位论文}}
  \fancyhead[CO]{\zihao{5} \nouppercase \leftmark}
  \fancyfoot[C]{{\zihao{-5} -~\thepage~-}}
  }
  \pagestyle{fancy}%%%%% 页眉 mzhy55
  \fancyhf{}
  \fancyhead[CE]{{\zihao{5} 燕山大学\CAST@value@degree 学位论文}}
  \fancyhead[CO]{{\zihao{5} \nouppercase \leftmark}}
  \fancyfoot[C]{{\zihao{-5} -~\thepage~-}}
\makeatother
%%%%%%% 以上内容不要修改!!! mzhy55

\chapter{程序代码的输入}
\label{appendixB}
% 重新定义 公式编号格式、图形、表格
\renewcommand\theequation{A-\arabic{equation}}
\renewcommand\thefigure{A-\arabic{figure}}
\renewcommand\thetable{A-\arabic{table}}

\begin{equation}\label{B1}
a=b^2+c_2.
\end{equation}

\section{插入程序代码}\label{appendixB-1}
\label{chap}

使用lisitings宏包可以在正文中插入程序的代码,插入的代码有自己的字体,可以实现行号、关键字高亮等功能。该环境的参数language决定了程序
的类型,例如\verb|language={[77]Fortran}|指定程序代码为FORTRAN语言;\verb|language={MATLAB}|指定程序代码为MATLAB的m语言。下边给出具体的例子。

\section{FORTRAN}\label{appendixB-2}

\begin{lstlisting}[language={[77]Fortran},
numbers=left,
numberstyle=\tiny,
basicstyle=\small\ttfamily,
stringstyle=\color{purple},
keywordstyle=\color{blue}\bfseries,
commentstyle=\color{brown},
frame=single]
C MATLAB gateway
      subroutine mexFunction(nlhs, plhs, nrhs, prhs)
C variables
      integer nlhs, nrhs
      integer plhs(*), prhs(*)
C input pointers
      pr_x=mxgetpr(prhs(1))
      pr_x1=mxgetpr(prhs(2))
C output pointers
      plhs(1)=mxCreateDoubleScalar(0)
      pr_y=mxGetPr(plhs(1))
C calculation
      call eim(%val(pr_x),%val(pr_x1),%val(pr_y))
      end subroutine mexFunction
\end{lstlisting}

\section{MATLAB}\label{appendixB-3}
\begin{verbatim}
\begin{lstlisting}[language={MATLAB},
numbers=left,
numberstyle=\tiny,
basicstyle=\small\ttfamily,
stringstyle=\color{purple},
keywordstyle=\color{blue}\bfseries,
commentstyle=\color{brown},
frame=single]
% bessel j

n=-0:0.1:12;
y=n*0;
b0n=besselj(0,n);
b1n=besselj(1,n);
plot(n,b0n,'-',n,b1n,'-',0:0.1:12,y)

ylabel('J_m(z)')
xlabel('z')
legend('J_0(z)','J_1(z)')
\end{lstlisting}
\end{verbatim}
输出的结果为:
\begin{lstlisting}[language={MATLAB},
numbers=left,
numberstyle=\tiny,
basicstyle=\small\ttfamily,
stringstyle=\color{purple},
keywordstyle=\color{blue}\bfseries,
commentstyle=\color{brown},
frame=single]
% bessel j

n=-0:0.1:12;
y=n*0;
b0n=besselj(0,n);
b1n=besselj(1,n);
plot(n,b0n,'-',n,b1n,'-',0:0.1:12,y)

ylabel('J_m(z)')
xlabel('z')
legend('J_0(z)','J_1(z)')
\end{lstlisting}

\section{C++}\label{appendixB-4}
\begin{verbatim}
\begin{lstlisting}[language={C++},
numbers=left,
numberstyle=\tiny,
basicstyle=\small\ttfamily,
stringstyle=\color{purple},
keywordstyle=\color{blue}\bfseries,
commentstyle=\color{brown},
frame=single]
# include{iostream.h}
void main()
int r;
double n;
{
cout<<"hello, LaTeX!"<<endl;
}
\end{lstlisting}
\end{verbatim}
输出的结果为:
\begin{lstlisting}[language={C++},
numbers=left,
numberstyle=\tiny,
basicstyle=\small\ttfamily,
stringstyle=\color{purple},
keywordstyle=\color{blue}\bfseries,
commentstyle=\color{brown},
frame=single]
# include{iostream.h}
void main()
int r;
double n;
{
cout<<"hello, LaTeX!"<<endl;
}
\end{lstlisting}
